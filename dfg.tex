\documentclass{scrartcl}

%This template is based on the DFG rtf form (see Readme on GitHub for last accessed date, and version in the header of the compiled PDF)
%
%Author: Martin Hoelzer, hoelzer.martin@gmail.com
%

\usepackage[utf8]{inputenc}
\usepackage[english]{proposal}

% final version?
\setboolean{finalcompile}{false}

% show DFG template version the current LaTeX template is based on in the header. Will be deactivated when 'finalcompile' is set 'true'
\ifthenelse{\boolean{finalcompile}}{}{\ihead*{DFG-form 53.01 - 03/24}}

% all config is done in Header.tex
\input{Header.tex}

% language dependent
% change labels, e.g.:
\crefname{figure}{Figure}{Figures}

%%%%%%%%%%%%%%%%%%%%%%%%%%%%%%%%%%%%%%%%%%%%%%%%%%%%%%%%%%%%%%%%%%%%%%%%%%%%%
%%%%  TITLE PAGE  %%%%%%%%%%%%%%%%%%%%%%%%%%%%%%%%%%%%%%%%%%%%%%%%%%%%%%%%%%%
%%%%%%%%%%%%%%%%%%%%%%%%%%%%%%%%%%%%%%%%%%%%%%%%%%%%%%%%%%%%%%%%%%%%%%%%%%%%%

\newcommand{\applicants}{Prof.Dr.\ Niels Sch\"utze, Technische Universität, Dresden}
\newcommand{\project}{[GDIE--Global Deficit Irrigation Experiment]}

% declare bibliography categories

% likely deprecated and these categories are not needed anymore
%\addtocategory{reviewed}{Hoelzer:16} 
%\nocite{}
%\addtocategory{nonreviewed}{Desiro:18} 
%\addtocategory{patents_pending}{} 
%\addtocategory{patents}{} 

\begin{document}
\pagestyle{empty}
\setcounter{page}{1}

%%%%%%%%%%%%%%%%%%%%%%%%%%%%%%%%%%%%%%%%%%%%%%%%%%%%%%%%%%%%%%%%%%%%%%%%%%%%%%
%% A NICE TITLE PAGE - OPTIONAL BC NOT IN THE ORIGINAL DFG TEMPLATE
% if you use the title page, set the \setcounter{page}{0} above!
%%%%%%%%%%%%%%%%%%%%%%%%%%%%%%%%%%%%%%%%%%%%%%%%%%%%%%%%%%%%%%%%%%%%%%%%%%%%%%
%\input{title_page}

%%%%%%%%%%%%%%%%%%%%%%%%%%%%%%%%%%%%%%%%%%%%%%%%%%%%%%%%%%%%%%%%%%%%%%%%%%%%%
%%%%  MAIN MATTER  %%%%%%%%%%%%%%%%%%%%%%%%%%%%%%%%%%%%%%%%%%%%%%%%%%%%%%%%%%
%%%%%%%%%%%%%%%%%%%%%%%%%%%%%%%%%%%%%%%%%%%%%%%%%%%%%%%%%%%%%%%%%%%%%%%%%%%%%
\cleardoublepage
\pagestyle{plain}

%%%%%%%%%%%%%%%%%%%%%%%%%%%%%%%%%%%%%%%%%%%%%%%%%%%%%%%%%%%%%%%%%%%%%%%%%%%%%
%%%% For first-time proposals, please also note the following: Please provide a
% concise description of your previous postgraduate training and research,
% highlighting your abilities and demonstrating that you are capable of carrying
% out the proposed project. Describe your scientific track record to date,
% mentioning research topics that you've worked on so far. Previous scientific
% accomplishments must not be directly related to the project.

%%%%%%%%%%%%%%%%%%%%%%%%%%%%%%%%%%%%%%%%%%%%%%%%%%%%%%%%%%%%%%%%%%%%%%%%%%%%%%%
%%%%  PROJECT DESCRIPTION - PROJECT PROPOSALS  %%%%%%%%%%%%%%%%%%%%%%%%%%%%%%%%
%%%%%%%%%%%%%%%%%%%%%%%%%%%%%%%%%%%%%%%%%%%%%%%%%%%%%%%%%%%%%%%%%%%%%%%%%%%%%%%
%% Sections 1-3 must not exceed 17 pages in total.
\todo{Sections 1--3 must not exceed 17 pages in total.}
{\raggedright{} \normalsize \bfseries 
	Project Description -- Project Proposals \par
	\applicants{} \par
	\project{} \par
	\rule{\textwidth}{0.5pt} \par
	%Project Description
}

%%%%%%%%%%%%%%%%%%%%%%%%%%%%%%%%%%%%%%%%%%%%%%%%%%%%%%%%%%%%%%%%%%%%%%%%%%%%%
%%%%  STATE OF THE ART AND PRELIMINARY WORK %%%%%%%%%%%%%%%%%%%%%%%%%%%%%%%%%
%%%%%%%%%%%%%%%%%%%%%%%%%%%%%%%%%%%%%%%%%%%%%%%%%%%%%%%%%%%%%%%%%%%%%%%%%%%%%
\section{Starting Point}
\label{sec:work-report}

\subsection*{State of the art and preliminary work}
% For renewal proposals, please report on your previous work. This report
% should also be understandable without referring to additional literature.

The agricultural sector is confronted with a profound challenge: it must produce more food and generate higher revenues while simultaneously reducing water consumption \citep{Doell2002,Rockstroem2010}. This challenge is exacerbated by the increasing scarcity of water resources, driven by factors such as climate change, population growth, and competing demands, making optimal irrigation management indispensable for sustainable agriculture \citep{Rosa2020}. Recent reviews highlight that water scarcity is now one of the greatest threats to global food security and agricultural productivity, necessitating innovative solutions for efficient water use.
Within this context, the improvement of crop water productivity within the framework of deficit irrigation (DI) has emerged as a key strategy for sustainable agriculture \citep{English1996}. A central aspect is the intraseasonal scheduling of irrigation under conditions of limited seasonal water availability. This requires the careful allocation of a finite water supply across multiple irrigation events throughout the crop growth cycle, taking into account crop-specific sensitivities to water deficits at different phenological stages \citep{Fereres2006,Vazifedoust2008}.

The adoption of advanced irrigation strategies, such as DI, has demonstrated the potential to enhance water productivity and stabilize crop yields under water-limited conditions \citep{English1996, Zhang1999}. By deliberately imposing controlled water deficits, DI enables the dual objectives of safeguarding agricultural production and conserving increasingly scarce water resources \citep{Schuetze2010}.

Unlike full irrigation, which aims to maximize yield under conditions of unlimited water supply, deficit irrigation (DI) necessitates the optimization of both the timing and amount of water application within the constraints of a predetermined, limited water budget. Thus, the resulting optimization problem is inherently high-dimensional. To address this complex optimization problem, both linear and non-linear programming approaches are employed using mathematical programming models \citep{Gorantiwar1995,Grove2018}. However, effective DI scheduling requires careful consideration of, e.g., crop-specific sensitivities to water deficits at various growth stages, as the impact of water stress can vary significantly throughout the crop cycle. In this context, crop models, such as AquaCrop–OSPy \citep{Kelly2021}, serve as valuable tools to complement DI experiments by simulating crop responses under varying, yet limited, water availability.

Consequently, simulation-based optimization has been widely used for optimizing DI systems, employing both open-loop and closed-loop control strategies for irrigation scheduling \citep{Schuetze2010}. Open-loop control refers to applying pre-determined irrigation schedules without using real-time feedback from the system, making it less adaptable to unexpected changes. In contrast, closed-loop control (or feedback control) continuously adjusts irrigation decisions based on real-time data from the field or a simulation model, allowing for dynamic adaptation to varying conditions. Numerous heuristic, evolutionary, and other global optimization approaches have been proposed to derive irrigation schedules that maximize crop water productivity, often demanding numerous experimental iterations to identify robust solutions \citep{Dantas2025}. However, the resulting schedules are typically suited for open-loop control strategies.

Recently, the application of closed-loop control strategies based on reinforcement learning (RL) has gained significant attention for irrigation scheduling \citep{Wang2021,Abioye2022,Bwambale2023, Umutoni2024, Ajith2025}. This trend is primarily driven by the rapid advancement of RL algorithms, which are now capable of tackling highly complex and dynamic tasks, such as AlphaZero’s mastery of strategic games like chess and Go \citep{Silver2018}. These new deep reinforcement (DRL) algorithms have demonstrated exceptional performance in real-world scenarios where continuous feedback and adaptation are essential, qualities that are equally critical for implementing closed-loop control in deficit irrigation (DI) systems.

Reinforcement learning is a general machine learning paradigm in which an agent seeks to maximize the cumulative reward it receives while interacting with a complex and uncertain environment \citep{Sutton1998}. In the context of irrigation systems, the environment may be a real-world irrigated field or a digital twin represented by a mathematical model, and the agent can correspond to the farmer or be implemented as a decision-making algorithm. The literature shows a growing number of RL studies focused on irrigation scheduling. However, these studies are often difficult to compare, as they differ in the models used, crop types, climate conditions, and RL implementation approaches \citep{Saikai2023}. This diversity is also reflected in the kinds of RL algorithms applied, including Temporal Difference (TD) Learning \citep{Schuetze2011}, Q-learning \citep{Yang2020,Chen2021,Alibabaei2022,Song2022,Devarajan2023,Ramli2024,QueirozPereira2025}, Policy Learning \citep{Alibabaei2022a,Ding2022,Saikai2023,Kelly2024,Agyeman2025,Abid2025}, and hybrid methods such as Actor-Critic Learning \citep{Alibabaei2022a,Heidari2025}.

Nevertheless, as reported by \citet{Saikai2023} and \citet{Kelly2024}, many RL implementations exhibit several shortcomings, including (i) the assumption of perfect information of future weather development, (ii) limited action spaces, (iii) simplified reward models, and (iv) the absence of explicit constraints, such as water availability. Although many RL studies generally aim to improve crop water productivity, only two papers explicitly implement volume limitations as a constraint in the sense of deficit irrigation \citep{Schuetze2011,Kelly2024}.

This study aims to advance the application of deep reinforcement learning (DRL) for deficit irrigation (DI) systems by benchmarking a novel DRL approach against existing strategies for closed-loop optimization of irrigation scheduling using the AquaCrop-OSPy model. The evaluation will be performed under varying levels of water scarcity, integrating detailed information about the irrigation system’s state and anticipated weather conditions. By incorporating these factors, the study seeks to deliver a comprehensive assessment of DRL’s effectiveness in optimizing irrigation practices, particularly under scenarios of limited water availability and changing climatic conditions.

%%%%%%%%%%%%%%%%%%%%%%%%%%%%%%%%%%%%%%%%%%%%%%%%%%%%%%%%%%%%%%%
% DEPRECATED since v54 (https://www.dfg.de/download/pdf/foerderung/programme/sachbeihilfe/info_vordrucksaenderungen.pdf)
% This became now part of the publication list in section 3, see below
%%%%%%%%%%%%%%%%%%%%%%%%%%%%%%%%%%%%%%%%%%%%%%%%%%%%%%%%%%%%%%%
%\subsection{Project-related publications}
%Please include a list of own publications that are related to the proposed project. 
%Sections 1.2.1 and 1.2.2 together must not exceed 10 publications; please number them consecutively.

%a maximum of ten publications
%\subsubsection{Articles published by outlets with scientific quality assurance, book publications, and works accepted for publication but not yet published}

% Own literature can be designated with a prefix, default 'O' for "Own". 
% If the function is not desired, delete the "labelprefix" commands directly below this comment 
% AND in the section: "Bibliography concerning the state of the art, the research objectives, and the work programme"
% Then, all your references should be numbered consecutively.
%\newrefcontext[labelprefix=O]
%\printbibliography[category=reviewed, heading=none, env=bibliographyNUM, resetnumbers]

%\subsubsection{Other publications, both peer-reviewed and non-peer-reviewed}
%\printbibliography[category=nonreviewed, heading=none, env=bibliographyNUM, resetnumbers=false]

%\subsubsection{Patents}

%\subsubsubsection{Pending}
%\printbibliography[category=patents_pending, heading=none]
%None.

%\subsubsubsection{Issued}
%\printbibliography[category=patents, heading=none]
%None.

%%%%%%%%%%%%%%%%%%%%%%%%%%%%%%%%%%%%%%%%%%%%%%%%%%%%%%%%%%%%%%%%%%%%%%%%%%%%%
%%%%  OBJECTIVES AND WP %%%%%%%% %%%%%%%%%%%%%%%%%%%%%%%%%%%%%%%%%%%%%%%%%%%%
%%%%%%%%%%%%%%%%%%%%%%%%%%%%%%%%%%%%%%%%%%%%%%%%%%%%%%%%%%%%%%%%%%%%%%%%%%%%%
\section{Objectives and work program}

\subsection{Anticipated total duration of the project}
Financial support is requested for \todo[inline]{three years.}

\subsection{Objectives}
\let\oldpara=\theparagraph
\addtocounter{secnumdepth}{1}
\renewcommand{\theparagraph}{Goal \arabic{paragraph}}

\paragraph{\textnormal{Goal one is ...}}

\subsection{Work program including proposed research methods}
%For each applicant
\todo[inline]{ca.\ 6-8 pages; ca. 4-8 WPs}

%%%%%%%%%%%%%%%%%%%%%%%%%%%%%%%%%%%%%%%%%%%%%%%%%%%%%%%%%%%%%%%%%%%%%%%% 
%%%%% BEGIN OF WORK PACKAGES %%%%%%%%%%%%%%%%%%%%%%%%%%%%%%%%%%%%%%%%%%%
%%%%%%%%%%%%%%%%%%%%%%%%%%%%%%%%%%%%%%%%%%%%%%%%%%%%%%%%%%%%%%%%%%%%%%%% 
%\titlespacing*{\paragraph}{0pt}{-0.2\baselineskip}{0.2\baselineskip}
\addtocounter{secnumdepth}{1}
%\newcommand\WP{WP\arabic{paragraph}}
\renewcommand{\theparagraph}{WP\arabic{paragraph}}
\RedeclareSectionCommand[beforeskip=-2.5ex plus-1ex minus -.2ex,%
afterskip=2.5ex plus 0.2ex minus 0.2ex]{paragraph}
%numbered sub-workpackages are supported - just use \subparagraph{}
\RedeclareSectionCommand[indent=0pt, beforeskip=-3.5ex plus-1ex minus -.2ex,%
afterskip=2.5ex plus 0.2ex minus 0.2ex]{subparagraph}
%%%%%%%%%%%%%%%%%%%%%%%%%%%%%%%%%%%%%%%%%%%%%%%%%%%%%%%%%%%%%%%%%%%%%%%%%%%%%

\hrulefill
\paragraph{First work package.}
\label{wp:1}

\input{wp/wp1.tex}

\hrulefill
\paragraph{Second work package.}
\label{wp:2}

\input{wp/wp2.tex}

%%%%%%%%%%%%%%%%%%%%%%%%%%%%%%%%%%%%%%%%%%%%%%%%%%%%%%%%%%%%%%%%%%%%%%%%%%%%%
%%%%  END OF WORK PACKAGES  %%%%%%%%%%%%%%%%%%%%%%%%%%%%%%%%%%%%%%%%%%%%%%%%%
%%%%%%%%%%%%%%%%%%%%%%%%%%%%%%%%%%%%%%%%%%%%%%%%%%%%%%%%%%%%%%%%%%%%%%%%%%%%%


%%%%%%%%%%%%%%%%%%%%%%%%%%%%%%%%%%%%%%%%%%%%%%%%%%%%%%%%%%%%%%%%%%%%%%%%%%%%%
%%%%  TIMELINE  %%%%%%%%%%%%%%%%%%%%%%%%%%%%%%%%%%%%%%%%%%%%%%%%%%%%%%%%%%%%%
%%%%%%%%%%%%%%%%%%%%%%%%%%%%%%%%%%%%%%%%%%%%%%%%%%%%%%%%%%%%%%%%%%%%%%%%%%%%%
%% This is not directly part of the DFG TRF template, but always helpful in my eyes
\let\theparagraph=\oldpara
\paragraph*{Time considerations}
\vspace{-0.5cm}
\begin{figure}[h]
	\centering
  	\resizebox{\textwidth}{!}{\includestandalone[mode=tex]{gantt/gantt}}
	\caption{Workpackage time considerations in months (M) and assigned person-months (PM) for Institute of ABC (ABC) and Institute for CDE (CDE)}
	\label{fig:timeline}
\end{figure}

\subsection{Handling of research data}
Data generated during this project will be used for scientific publications in
preprint and peer-reviewed journals. Therefore, all necessary data (such as raw
sequencing data) will be deposited in publicly available repositories
(\href{https://www.ncbi.nlm.nih.gov/geo/}{NCBI GEO} and
\href{https://www.ncbi.nlm.nih.gov/sra}{SRA}, \href{https://osf.io/}{OSF}). To
allow complete reproducibility of all analyses, the source code and executed
comands will be also made publicly available
(\href{https://github.com/}{GitHub}).  Thus, all relevant data will become
accessible for future use. 
\todo[inline]{FAIR principals}

\subsection{Relevance of sex, gender and/or diversity}
\todo[inline]{Text}

% Works cited from sections 1 and 2, both by the applicant(s) and by third parties. Please include DOI/URL if available. 
% A maximum of ten of your own works cited may be highlighted; font at least Arial 9 pt.
\section{Project- and subject-related list of publications}
\label{sec:bib}
\todo[inline]{Works cited from sections 1 and 2, both by the applicant(s) and by third parties. Please include DOI/URL if available. A maximum of ten of your own works cited may be highlighted; font at least Arial 9 pt.}
%\AtNextBibliography{\small}
\newrefcontext[labelprefix=]
\printbibliography[notcategory=reviewed, notcategory=nonreviewed, notcategory=patents_pending, notcategory=patents, heading=none, env=bibliographyNUM]	

\backmatter
%%%%%%%%%%%%%%%%%%%%%%%%%%%%%%%%%%%%%%%%%%%%%%%%%%%%%%%%%%%%%%%%%%%%%%%%%%%%%%%
%%%%  SUPPLEMENT  (new since 04/2022) %%%%%%%%%%%%%%%%%%%%%%%%%%%%%%%%%%%%%%%%%
%%%%%%%%%%%%%%%%%%%%%%%%%%%%%%%%%%%%%%%%%%%%%%%%%%%%%%%%%%%%%%%%%%%%%%%%%%%%%%%
\section{Supplementary information on the research context}
% Section 4 et seq. must not exceed 8 pages.

\subsection{Ethical and/or legal aspects of the project}

\subsubsection{General ethical aspects}
\todo[inline]{Text}

\subsubsection{Descriptions of proposed investigations involving humans, human materials or identifiable data}
\todo[inline]{Text}

\subsubsection{Descriptions of proposed investigations involving experiments on animals}
\todo[inline]{Text}

\subsubsection{Descriptions of projects involving genetic resources (or associated traditional knowledge) from a foreign country}
\todo[inline]{Text}

\subsubsection{Explanations regarding any possible safety-related aspects}

\subsubsubsection{\enquote{Dual Use Research of Concern}; foreign trade law}
\todo[inline]{Text}

\subsubsubsection{Risks in international cooperation}
\todo[inline]{Text}

\subsubsection{Considerations on aspects of ecological sustainability in the planning and implementation of the project}
\todo[inline]{Text}

%%%%%%%%%%%%%%%%%%%%%%%%%%%%%%%%%%%%%%%%%%%%%%%%%%%%%%%%%%%%%%%%%%%%%%%%%%%%%%%
%%%%  PEOPLE/COLLAB/FUNDING  %%%%%%%%%%%%%%%%%%%%%%%%%%%%%%%%%%%%%%%%%%%%%%%%%%%
%%%%%%%%%%%%%%%%%%%%%%%%%%%%%%%%%%%%%%%%%%%%%%%%%%%%%%%%%%%%%%%%%%%%%%%%%%%%%%%
\subsection{Employment status information}
%for each applicant, state the last name, first name, and employment status
%(including duration of contract and funding body, if on a fixed-term contract)
H\"olzer, Martin -- postdoctoral researcher, \todo[inline]{additional information}

\subsection{First-time proposal data}
%only if applicable: Last name, first name of first-time applicant
\todo[inline]{H\"olzer, Martin.}

\subsection{Composition of the project group}
%List only those individuals who will work on the project but will not be paid
%out of the project funds. State each person's name, academic title, employment
%status, and type of funding. 
The following individuals will work on the project but will not be paid out of
the project funds:

\begin{itemize}
\item Dude1
\item Dude2
\end{itemize}

\subsection{Researchers in Germany with whom you have agreed to cooperate on this project}
\todo[inline]{Text}

\subsection{Researchers abroad with whom you have agreed to cooperate on this project}
\todo[inline]{Text}

\subsection{Researchers with whom you have collaborated scientifically within the past three years}
% This information will help avoid potential conflicts of interest.
\begin{itemize}
  \item Prof.\ Dr.\ Dr.\ Foo Bar
\end{itemize}

\subsection{Project-relevant cooperation with commercial enterprises}
%If applicable, please note the EU guidelines on state aid or contract your research institution in this regard.
None.

\subsection{Project-relevant participation in commercial enterprises}
%Information on connections between the project and the production branch of the enterprise
None.

\subsection{Scientific equipment}
%List larger instruments that will be available to you for the project. These may include large computer facilities if computing capacity will be needed.
\todo[inline]{Text}
As something can be seen in \cref{fig:some_nice_graph}.

\begin{figure}
\centering
\missingfigure{schematics of a souvereign-class warp-reactor}
\caption{warp reactor}
\label{fig:some_nice_graph}
\end{figure}

\subsection{Other submissions}
%List any funding proposals for this project and/or major instrumentation previously submitted to a third party.
\todo[inline]{Text}

\subsection{Other information}
%Please use this section for any additional information you feel is relevant which has not been provided elsewhere.
In submitting a proposal to the DFG, I agree to adhere to the DFG's rules of good scientific practice and the \href{https://www.nature.com/articles/sdata201618}{FAIR principles}.

%%%%%%%%%%%%%%%%%%%%%%%%%%%%%%%%%%%%%%%%%%%%%%%%%%%%%%%%%%%%%%%%%%%%%%%%%%%%%%%
%%%%  REQUESTED MODULES/FUNDS  %%%%%%%%%%%%%%%%%%%%%%%%%%%%%%%%%%%%%%%%%%%%%%%%%%%
%%%%%%%%%%%%%%%%%%%%%%%%%%%%%%%%%%%%%%%%%%%%%%%%%%%%%%%%%%%%%%%%%%%%%%%%%%%%%%%
\section{Requested modules/funds}
% Explain each item for each applicant (stating last name, first name).

\subsection{Basic Module}

\subsubsection{Funding for Staff}
\begin{funds}[funding for staff]
The following staff positions are requested for \todo[inline]{xx} years each:

\positionmul{Research associate, TV-L 13, 36 months}{5375}{36}
\positionmul{Hiwi, TV-L 13, 12 months}{450}{12}

\end{funds}


\subsubsection{Direct Project Costs}
\begin{funds}[direct project costs]

\subsubsubsection{Equipment up to 10\,000\,\euro, Software and Consumables}

\position{Next-Generation Sequencing, Illumina HiSeq\,2500}{3000}

\subsubsubsection{Travel Expenses}
We apply for a total of 10\,000\,\euro\ for travel expenses.
\position{Travel}{10000}

\subsubsubsection{Visiting Researchers \textnormal{(excluding Mercator Fellows)}}
None.

\subsubsubsection{Expenses for Laboratory Animals}
None.

\subsubsubsection{Other Costs}
None.

\subsubsubsection{Project-related publication expenses}
We apply for a total of 1500\euro\ for publication expenses (750\,\euro\ per
year). Furthermore, we will submit articles to open access preprint repositories
such as \href{https://www.biorxiv.org/}{\textit{bioRxiv}}.
\position{Publication costs}{1500}

\end{funds}

\subsubsection{Instrumentation}

\subsubsubsection{Equipment exceeding 10\,000\,\euro}
Our laboratory is well equipped; all necessary instruments are available.

\subsubsubsection{Major Instrumentation exceeding 50\,000\,\euro}
Our laboratory is well equipped; all necessary instruments are available.


%%%%%%%%%%%%%%%%%%%%%%%%%%%%%%%%%%%%%%%%%%%%
%% The following are additional/other modules one might want to apply for
%% otherwise, just delete/comment
%%%%%%%%%%%%%%%%%%%%%%%%%%%%%%%%%%%%%%%%%%%%

\vspace*{2cm}\todo[inline]{The following are additional/other modules one might want to apply for. Otherwise, just delete/comment}

\subsection{Module Temporary Position for Principal Investigator}
\todo[inline]{Text}

\subsection{Module Replacement Funding}
\todo[inline]{Text}

\subsection{Module Temporary Clinician Substitute}
\todo[inline]{Text}

\subsection{Module Mercator Fellows}
\todo[inline]{Text}

\subsection{Module Workshop Funding}
\todo[inline]{Text}

\subsection{Module Public Relations}
\todo[inline]{Text}

\subsection{Module Standard Allowance for Gender Equality Measures}
\todo[inline]{Please detail what measures are planned to promote diversity and equal opportunities.  If you are submitting your proposal for an individual research grant within a network, note that this standard allowance may only be applied for within the coordination project. The coordination project must combine all such requests in its calculation.}

%%%%%%%%%%%%%%%%%%%%%%%%%%%%%%%%%%%%%%%%%%%%%%%%%%%%%%%%%%%%%%%%%%%%%%%%%%%%%%
%% SIGNATURE - OPTIONAL BC NOT IN THE ORIGINAL DFG TEMPLATE
%%%%%%%%%%%%%%%%%%%%%%%%%%%%%%%%%%%%%%%%%%%%%%%%%%%%%%%%%%%%%%%%%%%%%%%%%%%%%%
%\section{Signature}

%\includegraphics[width=0.18\textwidth]{/signatures/signature.pdf}\\
%Dr.\ Martin H\"olzer

%%%%%%%%%%%%%%%%%%%%%%%%%%%%%%%%%%%%%%%%%%%%%%%%%%%%%%%%%%%%%%%%%%%%%%%%%%%%%%
%% LoA - OPTIONAL BC NOT IN THE ORIGINAL DFG TEMPLATE
%%%%%%%%%%%%%%%%%%%%%%%%%%%%%%%%%%%%%%%%%%%%%%%%%%%%%%%%%%%%%%%%%%%%%%%%%%%%%%
%\section{List of attachments}
%\begin{itemize}
%  \item Curriculum Vit\ae, Dr.\ Martin H\"olzer
%\end{itemize}



\end{document}
